\chapter{译者序}
\pagestyle{empty}

第一次知道 Ada 这个程序语言是在大学读书时,那时对各种计算机语言感兴趣,在
图书馆里找各种语言类的书看。 Ada 进入了我的视野,她为我打开了另一扇窗。虽然
其后的工作中主要都是用 C,Ada 的一些精神也影响了我的 C 程序。一直觉得 Ada
没有被大多数的程序员接受而觉得有些遗憾,希望通过这个翻译可以引起更多人的
兴趣。

本文内容可自由用于个人或公司内部使用,但请不要未经许可出版发行。

由于翻译中难免有些词不达意的地方,欢迎各位读者指正。请把您的批评和建议发到
电邮: \url{zhu.qunying@gmail.com}。

\vspace{5em}
\begin{flushright}
朱群英

2009年于温哥华
\end{flushright}

%1974 年,美国国防部意识到花了太多的资金在软件上,特别是嵌入式系统,于是开始
%资助开发新的语言,经过一系列的努力,从开始的好几个阶段的需求定义到招标及
%并行的开发和评估,在 1983 年成为了 ANSI 美国国家标准。而后在 1987 年又成为
%ISO 国际标准。之后经过了 Ada 95,Ada 2005 的标准化进程,每一次的标准制定,
%都为 Ada 带来了新的功能,同时保有她原本的特色。

%Ada 从孕育到现在,已经有30多个年头了。尽管她有许多优点,Ada 的使用
%范围并不广泛。然而在两个领域里她占有绝对的优势,一个是在大型,复杂及长期
%使用的系统,如空中管制系统,地铁系统,另一个是要求极端安全的实时系统,如
%飞机和汽车中使用的系统。

