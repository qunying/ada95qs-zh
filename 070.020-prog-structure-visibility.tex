% -*- coding: utf-8 -*-
% Ada 95 品质和风格
% 版权所有 (C) 2011 朱群英
% Copyright (C) 2011 Zhu Qun-Ying
%
% 本书的 TeX 代码和由之生成的 ps,pdf,html,等其他格式的文件
% 遵循 GNU通用公共授权第三版或其后的版本。
%
% 本书是基于有益的目的而加以发布,然而不负任何担保责任。
%
% 您应已收到附随于本书的GNU通用公共授权的副本;如果没有,
% 请参考 <http://www.gnu.org/licenses/gpl.html>

\section{可见度}
Ada 的可见度控制功能是Ada语言最重要的优势之一, 特别是当``分开开发一个大型系统的
各部分''时。她强制信息的隐藏和分离利害关系的能力就来源于此。
假如没有这些功能,而过度依赖于\texttt{use}语句,是很浪费而危险的。
参见准则\ref{c:prog-prac:data:visibility} 和
 \ref{c:oof:tagged:visibility:derived-tag}。

\subsection{接口的最小化}
\label{prog-struct:visiblity:min-intf}
\glentry{准则}
\begin{itemize}
\item 只把使用包需要的信息放在规范中。
\item 最小化包规范中的声明数量。
\item 不要因为容易建立而包含额外的操作。
\item 在包规范中,最小化语境语句 (\texttt{with})。
\item 重新考虑有大量参数的子程序。
\item 不要为了限制参数的数量而在子程序或包中操纵全局数据。
\item 避免不必要的可见度,对用户隐藏程序的具体实现。
\item 用子库单元来控制子系统接口的可见度。
\item 用私有子库单元来声明外部用不到的信息。
\item 用子库单元来对不同客户呈现某一实体的不同视点。
\item 在找出各样预期客户的接口逻辑后,设计(及重设计) 接口。
\end{itemize}


\glentry{范例}
\begin{lstlisting}
-------------------------------------------------------------------------
package Telephone_Book is
   type Listing is limited private;
   procedure Set_Name (New_Name : in     String;
                       Current  : in out Listing);
   procedure Insert (Name    : in     String;
                     Current : in out Listing);
   procedure Delete (Obsolete : in     String;
                     Current  : in out Listing);
private
   type Information;
   type Listing is access Information;
end Telephone_Book;
-------------------------------------------------------------------------
package body Telephone_Book is
   -- Full details of record for a listing
   type Information is
      record
         ...
         Next : Listing;
      end record;
   First : Listing;
   procedure Set_Name (New_Name : in     String;
                       Current  : in out Listing) is separate;
   procedure Insert (Name    : in      String;
                     Current : in out  Listing) is separate;
   procedure Delete (Obsolete : in     String;
                     Current  : in out Listing) is separate;
end Telephone_Book;
-------------------------------------------------------------------------
\end{lstlisting}

\glentry{原理}
\begin{blockindent}
规范中的每一个实体,都要认真考虑其转移到子包或父包实体中的可能。
越少的冗余细节,程序、包或子程序就越容易理解。对于维护者而言,知道
准确的包接口很重要,以便于理解改动的效果。子程序的接口不仅限于参数。
任何在包或子程序中修改全局变量的行为,都可被外界视为未公开的接口。
\end{blockindent}
