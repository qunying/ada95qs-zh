% -*- coding: utf-8 -*-
% Ada 95 品质和风格
% 版权所有 (C) 2011 朱群英
% Copyright (C) 2011 Zhu Qun-Ying
%
% 本书的 TeX 代码和由之生成的 ps,pdf,html,等其他格式的文件
% 遵循 GNU通用公共授权第三版或其后的版本。
%
% 本书是基于有益的目的而加以发布,然而不负任何担保责任。
%
% 您应已收到附随于本书的GNU通用公共授权的副本;如果没有,
% 请参考 <http://www.gnu.org/licenses/gpl.html>

\section{总结}
\glentry{高级结构}
\begin{itemize}
\item 把每一个库单元的规约放在独立于实体的文件中。
\item 避免定义不是主程序的库单元子程序。如果定义了这样的子程序,
对每一个库单元子程序创建一个明确的单独的规约文件。
\item 把亚单元的使用降到最低。
\item 对于亚单元而言,用子库单元来把一个子系统构建为可管理的多个单元。
\item 每一个亚单元存放在独立的文件中。
\item 使用一致的文件命名约定。
\item 对于包主体的嵌套,用私有子包并在父包中引用。
\item 对于(其他)子单元用于扩展父单元的抽象或服务需要的数据和子程序,
 用私有子单元规约。

\end{itemize}
