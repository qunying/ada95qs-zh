% -*- coding: utf-8 -*-
% Ada 95 品质和风格
% 版权所有 (C) 2009,2010, 朱群英
% Copyright (C) 2009,2010, Zhu Qun-Ying
%
% 本书的 TeX 代码和由之生成的 ps,pdf,html,等其他格式的文件
% 遵循 GNU通用公共授权第三版或其后的版本。
%
% 本书是基于有益的目的而加以发布,然而不负任何担保责任。
%
% 您应已收到附随于本书的GNU通用公共授权的副本;如果没有,
% 请参考 <http://www.gnu.org/licenses/gpl.html>

\section{使用类型}

强类型促进软件的可靠性。一个对象的类型定义了所有合法的值和操作,
这允许编译器在编译时检验和识别潜在的错误。
另外,类型的规则允许编译器生成在执行时
检验违反类型约束的代码。运用这些 Ada 编译器的特性,比其他没那么强
类型的语言更早以及更完整的检测出错误。

\subsection{类型声明准则}
\glentry{准则}
\begin{itemize}
    \item 尽量限制标量类型的值域。
    \item 从应用中查找可能值的信息。
    \item 不要重用标准 (Standard) 包中的子类型名称。
    \item 用子类型声明来促进程序的可读性 (\cite{booch87})。
    \item 协调使用导出类型和子类型 (\ref{c:prog-prac:types:derived})。
\end{itemize}

\glentry{范例}
\begin{blockindent}
\begin{lstlisting}
subtype Card_Image is String (1 .. 80);
Input_Line : Card_Image := (others => ' ');
-- restricted integer type:
type    Day_Of_Leap_Year     is                  range 1 .. 366;
subtype Day_Of_Non_Leap_Year is Day_Of_Leap_Year range 1 .. 365;
\end{lstlisting}

下面的声明,程序员的意思是:``我不知道有多少。'' 但是实际的基本值域会
掩埋在代码中或作为系统参数:

\begin{lstlisting}
Employee_Count : Integer;
\end{lstlisting}
\end{blockindent}

\glentry{原理}
\begin{blockindent}
把无意义的数值从合法的范围内剔除,增加了编译器检测到错误的能力。
这也提高了程序的可读性。另外,当对象声明为这些子类型时,
这也让你仔细考虑每次对象的使用。

不同的实现对于大多数预定义类型提供了不同的一套数值。一个读者不能从预定义
的名字知道预期的值域。当预定义名字被超载这样的情形会更恶化。
\end{blockindent}
