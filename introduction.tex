\chapter{介绍}
\label{c:intro}
\pagenumbering{arabic}
\pagestyle{fancy}

风格这个写作的重要属性往往被忽略,而写作的风格直接影响最后作品的可读性和
可理解性。 编程的风格,即使用某个电脑编程语言书写源程序,也受到相当的忽视。
程序不仅仅需要被机器读懂,还要被人读并易于理解。这个需求对生产高质量的产品
同样重要,即在预算内按时按质完成符合客户要求的产品。本书试图帮助电脑专业人员
写出更好的 Ada 程序。本书呈现给读者一套有针对性的风格准则,从而 Ada 95 的
强大功能可以在遵守一些规则下使用。

每个\emph{准则}都简明叙述了应遵守的规则以及从\emph{原理}上解释其原因。
大多数情况下,都会有一个使用该准则的\emph{范例},某些情况下,还会有进一步
的范例来显示不遵守准则的后果。所有可能的使用上的\emph{特殊情况}都会特
别指出,如果可能,会有进一步的\emph{注解}。某些情况下,会使用一个
\emph{具体事例}来进一步呈现更加精确的准则,从而可以作为标准实施。某些特
别选出来的情况,\emph{自动化注解}会探讨怎样自动实施这个准则。

Ada 当初被设计为支持开发高质量、可靠、可重用、可移植的软件。由于各种原因,
没有程序语言可以只凭自己本身来保证达到这些目的。例如,编程必须嵌入在一个
有序的开发流程,从而分析需求,设计,实现,查验,确认和维护得以井然有序。
从获得好评的软件工程规范中总结出的好的编程实践在语言的使用中必须遵守。
本书试图在工程规范和 Ada 实际编程中的差异架起桥梁。

本书中的许多准则都为促进源代码的清晰而设。这些准则的目标是提高程序的演变、
改写和维护的容易度。可理解的源代码,使其看起来正确和可靠的可能性更高。
代码的改写要求对程序有完整的理解,清晰的代码对于理解有相当的帮助。有效的
代码改写是代码重用不可缺的条件, 而代码的重用有大幅降低系统开发成本的潜力。
最后,系统使用期间的维护(实际上是演变)是个持续的昂贵过程, 清晰的代码在维持
较低的维护成本上起了主要作用。 在整个系统的生命周期中,代码被阅读和理解的次数
大大多于编写,所以投资在代码的可读性和可理解性上是值得的。

在本章剩下的小节中,总结了一下本书的结构和如何把呈现的内容给不同的角色使用,
如新的 Ada 程序员,有经验的 Ada 程序员,面向对象的程序员,项目经理,承包商,
标准制定组织,从现有的 Ada 83 程序过渡到 Ada 95 的计划者。
 
\section{本书的结构}
本书的格式遵循受到广泛欢迎的《Ada 品质和风格:专业程序员的准则》版本号
 02.01.01 (AQ\&S 83) (软件生产力协会 1992) 所用的格式。本指南根据程序员
为了写出高质量、可靠、可重用、可移植的 Ada 程序,所做出的各个重要决定,
进行分章描述。章节中某些部分会有所重复,因为不是所有的决定都可以独立做出。

本书对源代码的呈现、可读性、程序结构、并行处理、可移植性、可重用性、
性能以及面向对象进行了分章阐述。 每一章都以一个本章准则的\emph{摘要}来结束。
最后一章呈现了一个完整的哲学家吃晚餐的实现,这个范例由 Michael B. Feldman 博士
和 Bjorn Kallberg 先生提供。这个范例中使用了本书中的许多准则。附录给出了一个
Ada 参考手册 (1995) 和本书中的准则之间的交互参照矩阵。

本书的用词是过去 20 年软件工程中发展出的通用词语。软件工程是个快速发展的领域,
有许多相对新的概念和术语。为了有个共通的参考框架,需要的定义会从 Ada 参考手册
(1995) 和 原理 (1995) 中取用。

本书多处引用了其他有关 Ada 风格和问题的文献和资料。相关的参考资料列表在书
的末尾。数模也提供了参考文献列表。

``Ada'' 这个专有名词在本书中是指在 1995 年 二月公布的最新 Ada 标准
\footnote{译者注:现在最新的标准是 Ada 2005 了。} (有时也用 Ada 95 来表示)。
引用早前的 Ada 标准都会清楚的注明``Ada 83''。

\subsection{源代码的呈现和可读性}
第\ref{c:src-presentation}章和第\ref{c:readability}章直接针对解决创作清晰、
可读、可理解源代码中的问题。第\ref{c:src-presentation}章集中讨论代码的格式,
第\ref{c:readability}章解决运用注释、命名规则和类型的问题。
