% -*- coding: utf-8 -*-
% Ada 95 品质和风格
% 版权所有 (C) 2009, 朱群英
% Copyright (C) 2009, Zhu Qun-Ying
%
% 本书的 TeX 代码和由之生成的 ps,pdf,html,等其他格式的文件
% 遵循 GNU通用公共授权第三版或其后的版本。
%
% 本书是基于有益的目的而加以发布,然而不负任何担保责任。
%
% 您应已收到附随于本书的GNU通用公共授权的副本;如果没有,
% 请参考 <http://www.gnu.org/licenses/gpl.html>

\cleardoublepage
\addcontentsline{toc}{chapter}{文献}
\begin{thebibliography}{widest-label}\label{bib}

\bibitem[ARM 1983]{arm83} 《Reference Manual for the Ada Programming
Language》. Department of Defense, ANSI/MIL-STD-1815A.

\bibitem[ARM 1995]{arm95} 《Ada 95 Reference Manual》,
ISO/8652-1995 Cambridge, Massachusetts: Intermetrics, Inc.

\bibitem[Barnes 1989]{barnes89} Barnes, J.G.P.《Programming in Ada》,3d ed.
Reading, Massachusetts: Addison-Wesley.

\bibitem[Barnes 1996]{barnes96} Barnes, J.G.P.《Programming in Ada 95》,
Reading, Massachusetts: Addison-Wesley.

\bibitem[Cohen 1986]{cohen86} Cohen, N.H.,《Ada as a Second Language》.
 New York, New York: McGraw-Hill Inc.

\bibitem[Cohen 1996]{cohen96} Cohen, N.H.,《Ada as a Second Language》, 2nd edition, New York, New York: McGraw-Hill Inc.

\bibitem[原理 1995]{rational95} 《Ada 95 Rationale》, Cambridge,
Massachusetts: Intermetrics, Inc. 

\end{thebibliography}
