% -*- coding: utf-8 -*-
% Ada 95 品质和风格
% 版权所有 (C) 2011 朱群英
% Copyright (C) 2011 Zhu Qun-Ying
%
% 本书的 TeX 代码和由之生成的 ps,pdf,html,等其他格式的文件
% 遵循 GNU通用公共授权第三版或其后的版本。
%
% 本书是基于有益的目的而加以发布,然而不负任何担保责任。
%
% 您应已收到附随于本书的GNU通用公共授权的副本;如果没有,
% 请参考 <http://www.gnu.org/licenses/gpl.html>

\section{高级结构}
\begin{blockindent}
构建良好的程序很同意被理解、提高和维护。构建糟糕的程序在维护时,
经常需要重新构建以方便工作。下面列出的许多准则,是作为通用程序设计准则
给出的。
\end{blockindent}

\subsection{分别编译能力}
\glentry{准则}
\begin{itemize}
\item 把每一个库单元的规范放在独立于实体的文件中。
\item 避免定义不是主程序的库单元子程序。如果定义了这样的子程序,
对每一个库单元子程序创建一个明确的单独的规范文件。
\item 把子单元的使用降到最低。
\item 对于子单元,用子库单元来把一个子系统构建为可管理的多个单元。
\item 每一个子单元存放在独立的文件中。
\item 使用一致的文件命名约定。
\item 对于包体的嵌套,用私有子包并在父包中引用。
\item 对于(其他)子单元用于扩展父单元的抽象或服务需要的数据和子程序,
 用私有子单元规范。
\end{itemize}

\glentry{范例}
\begin{blockindent}
下面文件名说明了一种可能的文件组织以及相随的一致命名约定。
\end{blockindent}
