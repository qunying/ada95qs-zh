% -*- coding: utf-8 -*-
% Ada 95 品质和风格
% 版权所有 (C) 2011 朱群英
% Copyright (C) 2011 Zhu Qun-Ying
%
% 本书的 TeX 代码和由之生成的 ps,pdf,html,等其他格式的文件
% 遵循 GNU通用公共授权第三版或其后的版本。
%
% 本书是基于有益的目的而加以发布,然而不负任何担保责任。
%
% 您应已收到附随于本书的GNU通用公共授权的副本;如果没有,
% 请参考 <http://www.gnu.org/licenses/gpl.html>

\section{高级结构}
\begin{blockindent}
构建良好的程序很同意被理解、提高和维护。构建糟糕的程序在维护时,
经常需要重新构建以方便工作。下面列出的许多准则,是作为通用程序设计准则
给出的。
\end{blockindent}

\subsection{分别编译能力}
\glentry{准则}
\begin{itemize}
\item 把每一个库单元的规约放在独立于实体的文件中。
\item 避免定义不是主程序的库单元子程序。如果定义了这样的子程序,
对每一个库单元子程序创建一个明确的单独的规约文件。
\item 把子单元的使用降到最低。
\item 对于子单元,用子库单元来把一个子系统构建为可管理的多个单元。
\item 每一个子单元存放在独立的文件中。
\item 使用一致的文件命名约定。
\item 对于包体的嵌套,用私有子包并在父包中引用。
\item 对于(其他)子单元用于扩展父单元的抽象或服务需要的数据和子程序,
 用私有子单元规约。
\end{itemize}

\glentry{范例}
\begin{blockindent}
下面的文件名说明了一种可能的文件组织以及相随的一致命名约定。
库单元的名字用后缀 adb 来表示单元主体,后缀 ads 表示单元规约,
任何其他包含子单元的文件的名字,由体名字和子单元名字以下划线
分隔:
\begin{lstlisting}
text_io.ads                 -- the specification
text_io.adb                 -- the body
text_io_integer_io.adb      -- a subunit
text_io_fixed_io.adb        -- a subunit
text_io_float_io.adb        -- a subunit
text_io_enumeration_io.adb  -- a subunit
\end{lstlisting}

依赖于文件系统允许文件名中可以用什么字符,你可以在文件
名字中用更清晰的方式来区分父和子单元的名字。
如果文件系统允许使用 \texttt{"#"} 字符,那可以用 \texttt{#}
来分隔主体和子单元的名字:
\begin{lstlisting}
text_io.ads                 -- the specification
text_io.adb                 -- the body
text_io#integer_io.adb      -- a subunit
text_io#fixed_io.adb        -- a subunit
text_io#float_io.adb        -- a subunit
text_io#enumeration_io.adb  -- a subunit
\end{lstlisting}

有些操作系统时区分大小写的,尽管 Ada 本身是不区分大小写的语言。
这时,可以选择全用小写来表示文件名。
\end{blockindent}

\glentry{原理}
\begin{blockindent}
本准则侧重在分散文件上的主要原因,是最小化每次改动后需要重新编译的
文件数。通常,在软件开发过程中,主体的改动要比规约要频繁得多。
如果规约和主体在同一个文件中,那么每次主体的编译也会编译一次规约,
虽然规约并没有改变。 因为规约定义了单元和其所有使用者之间的接口,
每次规约的重编译都会导致所有使用者必须重编译以确认符合规约。如果
使用者的规约和主体也在一个文件里,那任何使用这些单元的使用者也要
重编译,以此类推。这个小变化,会强制大量本可以避免的重编译,严重
拖慢了项目的开发和测试阶段。这就是为什么你要把所有库单元 (非嵌套单元)
中的规约和其主体分别存放在不同的文件中。

库单元中子程序要尽量短小。库单元子程序的真正运用只有作为主子程序时。
几乎在其他任何情况下,最好时把子程序嵌套在包中。这为子程序在主体中
提供了一个本地化所需数据的场所。而且,这也减低了系统中单独模块的数量。
\end{blockindent}
