% -*- coding: utf-8 -*-
% Ada 95 品质和风格
% 版权所有 (C) 2009, 朱群英
% Copyright (C) 2009, Zhu Qun-Ying
%
% 本书的 TeX 代码和由之生成的 ps,pdf,html,等其他格式的文件
% 遵循 GNU通用公共授权第三版或其后的版本。
%
% 本书是基于有益的目的而加以发布,然而不负任何担保责任。
%
% 您应已收到附随于本书的GNU通用公共授权的副本;如果没有,
% 请参考 <http://www.gnu.org/licenses/gpl.html>

\chapter{可读性}
\label{c:readability}

本章推荐如何使用Ada的功能,使得代码更易阅读和理解。有许多有关注释和可读性的
迷思。命名和代码结构比注释对真正的可读性承担更多的责任。拥有和代码一样多
的注释并不代表可读性;它更可能是作者不理解什么是需要沟通的重要部分。

\section{拼写}
源代码中的拼写规则包括大写、下划线的应用、数字和缩写。如果你一致遵循这些规则,
那写出来的代码就会更清晰和易读。

\subsection{下划线的应用}
\glentry{准则}
\begin{itemize}
    \item 用下划线分隔复合名字中的词。
\end{itemize}

\glentry{范例}
\begin{blockindent}
\noindent
\begin{lstlisting}
Miles_Per_Hour
Entry_Value
\end{lstlisting}
\end{blockindent}

\glentry{原理}
\begin{blockindent}
当一个标识符包括多个词,如果用下划线分隔词,那会更容易阅读。事实上,
英语中也有这样的惯例,复合词用练字符或空格分隔。另外,为了促进代码的
可读性,如果在名字中用下划线,代码格式器对于大小写有更多的控制。
参见准则\ref{s:readability:cap}。
\end{blockindent}

\subsection{数字}
\glentry{准则}
\begin{itemize}
    \item 使用一致的风格来呈现数字。
    \item 使用对问题适当的根植来呈现字面值。
    \item 使用下划线来分隔数字,就像在普通文字中用逗号或句号
(或对于非十进制用空格)。
    \item 当使用科学标示法,用一致的大写或小写的E。
    \item 在非十进制数字中,要么使用全大写或小写来标示其中的字母。
\end{itemize}

\glentry{具体事例}
\begin{blockindent}
\noindent
\begin{itemize}
    \item[-] 十进制或八进制数字,在小数点的左边,每三个数字一组,在右边,
每五个数字一组。
    \item[-] 科学标示中,总用大写字母E。
    \item[-] 用大写字母表示在十以上进制的数字。
    \item[-] 十六进制数字,小数点两边都是每四个数字一组。
\end{itemize}
\end{blockindent}

\glentry{范例}
\begin{blockindent}
\noindent
\begin{lstlisting}
type Maximum_Samples     is range          1 .. 1_000_000;
type Legal_Hex_Address   is range   16#0000# ..   16#FFFF#;
type Legal_Octal_Address is range 8#000_000# .. 8#777_777#;

Avogadro_Number : constant := 6.02216_9E+23;
\end{lstlisting}
用一下方式表示常量$1/3$:
\begin{lstlisting}
One_Third : constant := 1.0 / 3.0;
\end{lstlisting}
或
\begin{lstlisting}
One_Third_Base_3 : constant := 3#0.1#;
\end{lstlisting}
不要用:
\begin{lstlisting}
One_Third_As_Decimal_Approximation : constant := 0.33333_33333_3333;
\end{lstlisting}
\end{blockindent}

\glentry{原理}
\begin{blockindent}
大写或小写字母的一致使用,有助于搜索数字。用下划线来把数字组成熟悉的模式。
与平时使用模式的一致是可读性的一大要素。
\end{blockindent}

\glentry{注解}
\begin{blockindent}
如果一个分数用一个它可以终止的基数进制来表示,而不用一个重复的表示法,就象
上面例子中的\texttt{3\#0.1\#},在转换到机器的基数时,有可能更精确。
\end{blockindent}

\subsection{大写}
\label{s:readability:cap}
\glentry{准则}
\begin{itemize}
    \item 让保留字和程序的其他组件从视觉上分开。
\end{itemize}
\glentry{具体事例}
\begin{blockindent}
\noindent
\begin{itemize}
    \item[-] 所有保留字用小写 (当用做保留字时)。
    \item[-] 所有其他标识符混用大小写,每个词用一个大写字母开始,
并用下划线分开。
    \item[-] 用大写字母表示缩略语或首写字母缩写 (见自动化注释)。
\end{itemize}
\end{blockindent}
\glentry{范例}
\begin{blockindent}
\noindent
\begin{lstlisting}
...

type Second_Of_Day      is range 0 .. 86_400;
type Noon_Relative_Time is (Before_Noon, After_Noon, High_Noon);

subtype Morning   is Second_Of_Day range 0 .. 86_400 / 2 - 1;
subtype Afternoon is Second_Of_Day range Morning'Last + 2 .. 86_400;

...

Current_Time := Second_Of_Day(Calendar.Seconds(Calendar.Clock));

if Current_Time in Morning then
    Time_Of_Day := Before_Noon;
elsif Current_Time in Afternoon then
    Time_Of_Day := After_Noon;
else
    Time_Of_Day := High_Noon;
end if;

case Time_Of_Day is
    when Before_Noon =>   Get_Ready_For_Lunch;
    when High_Noon   =>   Eat_Lunch;
    when After_Noon  =>   Get_To_Work;
end case;

...
\end{lstlisting}
\end{blockindent}

\section{总结}
\glentry{拼写}
\noindent
\begin{itemize}
    \item 用下划线分隔复合名字中的词。
    \item 使用一致的风格来呈现数字。
    \item 使用对问题适当的根植来呈现字面值。
    \item 使用下划线来分隔数字,就像在普通文字中用逗号或句号
(或对于非十进制用空格)。
    \item 当使用科学标示法,用一致的大写或小写的E。
    \item 在非十进制数字中,要么使用全大写或小写来标示其中的字母。
    \item 让保留字和程序的其他组件从视觉上分开。
\end{itemize}

