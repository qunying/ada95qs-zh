% -*- coding: utf-8 -*-
% Ada 95 品质和风格
% 版权所有 (C) 2009, 朱群英
% Copyright (C) 2009, Zhu Qun-Ying
%
% 本书的 TeX 代码和由之生成的 ps,pdf,html,等其他格式的文件
% 遵循 GNU通用公共授权第三版或其后的版本。
%
% 本书是基于有益的目的而加以发布,然而不负任何担保责任。
%
% 您应已收到附随于本书的GNU通用公共授权的副本;如果没有,
% 请参考 <http://www.gnu.org/licenses/gpl.html>

\chapter{源代码的呈现}
\label{c:src}
代码在纸张和屏幕上的版面规划对于代码的可读性起了很大的作用。本章包含了
令代码更可读的呈现准则。

在通用的准则之外,在\emph{实例}部分,还有特别的推荐。如果你不同意这些特别
的推荐,你可以采用自己的规范,同时也遵守了准则。总之,保持全项目的一致。

完全一致的版面规范很难靠人工去完成和检查,所以,对于版面规范的自动化,
你可能比较细欢使用工具根据不同参数来格式化代码,或者把准则整合到自动代码
模块。某些准则和特别的推荐不能用自动格式化工具完成,因为它们是基于 Ada 语句
的语义而非语法。\emph{自动化注解}会给出更多的详细内容。

\section{代码格式化}
代码的格式化影响代码的观感,而不是代码的功用。这里讨论的问题包括:水平间隔、
缩进、对齐、分页和行长度。最重要的准则是在编译单元和整个项目中保持一致。

\subsection{水平间隔}
\glemph{准则}
\begin{itemize}
    \item 分割符之间使用固定的间隔。
    \item 使用和写一般文章一样的空格。
\end{itemize} 

\glemph{实例}
\begin{itemize}%用 itemize 来达到缩进的目的,有更好的方法吗?
    \item[] 特别指出,在下列地方,保留至少一个空格,本书中的范例也使用
一样的准则。紧跟的准则会为了对齐的原因,可能需要更多的竖向空间。
    \begin{itemize}
	\item[-] 在下列分隔符和二进制操作符的前后:
	    \begin{quote}
	    \texttt{
	    \begin{tabular}[h]{l l l l l l l l l l l l}
		+  & & -  & & * & & /  & & \& \\
		<  & & =  & & > & & /= & & <= & >=\\
		:= & & => & & | & & .. \\
		: \\
		<>\\
	    \end{tabular}
	    }
	    \end{quote}
    \end{itemize}
\end{itemize}
